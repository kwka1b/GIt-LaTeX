\documentclass{ctexart}
\usepackage{graphicx} % Required for inserting images
\usepackage{geometry}
\usepackage{fancyhdr}
\usepackage{lastpage}

\geometry{left=2cm,right=2cm,top=3.5cm,bottom=3.5cm}
\pagestyle{fancy}
\fancyhf{}
\lhead{23020007051}
\chead{中国海洋大学}
\rhead{孔维凯}
\lfoot{\LaTeX{}}
\cfoot{第 \thepage 页(共\pageref{LastPage}页)}
\rfoot{\today}

\title{\textbf{Git和LaTeX课程实例报告}}
\author{\textbf{姓名:孔维凯\ \ \ \ \ \ 学号:23020007051}}
\date{\underline{\textbf{\today}}}

\begin{document}

\maketitle

\newpage

\tableofcontents

\newpage

\section{Git}
\subsection{实例1}
\subsubsection{练习内容}
下载并使用他人的代码
\subsubsection{结果}
首先运用git init bare命令初始化一个空仓库,仓库名为bare,再输入cd+bare文件夹地址,然后使用git clone+代码地址将代码文件下载到bare文件夹里,就可以使用了。
\subsubsection{解题感悟}
通过这种方式就可以使用他人的代码,省时省力,但是使用前注意要查看协议,不能违反协议使用代码。
\subsection{实例2}
\subsubsection{练习内容}
将git和github账户连接到一起
\subsubsection{结果}
输入ssh-keygen -t rsa -b 4096 -C "your\ email@example.com"其中your\ email@example.com改为你的电子邮箱地址,然后输入密钥存储地址,设置密码,然后打开密钥文本复制密钥在github账户上添加SSH密钥后就可以了。
\subsubsection{解题感悟}
将git和github账户链接在一起便于使用git从github开源网站上下载他人的代码,便于使用。
\subsection{实例3}
\subsubsection{练习内容}
修改下载文件的地址
\subsubsection{结果}
使用cd+文件夹地址就可以将当前默认地址修改为预期地址
\subsubsection{解题感悟}
这样做方便使用者快捷查看下载的代码。
\subsection{实例4}
\subsubsection{练习内容}
合并分支、发送合并文件以及删除分支
\subsubsection{结果}
git checkout master

git merge +分支名可以合并分支

git push+<远程仓库名> <本地分支名>:<远程分支名>就可以将合并后的更改到远程仓库

git branch -d +分支名可以删除本地分支
\subsubsection{解题感悟}
git的指令可以方便用户按照料想快捷完成操作。
\subsection{实例5}
\subsubsection{练习内容}
如何查看基本操作指令和git版本
\subsubsection{结果}
输入git help指令即可查看一些基本操作指令,通过输入git version指令可以查看当前git版本
\subsubsection{解题感悟}
通过help指令可以快速学习一些基本操作指令,通过version查看版本情况时刻保持版本更新。
\subsection{实例6}
\subsubsection{练习内容}
设置全局用户名和邮箱‌
\subsubsection{结果}
通过git config --global user.name+名字来设置全局用户名

通过git config --global user.email+电子邮件地址来设置全局邮箱

通过git config --list指令来显示参数
\subsubsection{解题感悟}
这些指令可以帮助识别代码的作者。
\subsection{实例7}
\subsubsection{练习内容}
分支相关指令
\subsubsection{结果}
git checkout+分支名用于切换分支

git checkout -b+分支名用于创建并切换到新分支

git branch用于查看git仓库的分支情况
\subsubsection{解题感悟}
分支让开发者可以在不影响主分支的情况下进行试验,有利于保护原有代码。
\subsection{实例8}
\subsubsection{练习内容}
提交代码覆盖了别人的代码怎么挽回
\subsubsection{结果}
首先使用git log命令找到覆盖之前的提交,用git checkout语句检出那个安全的提交,然后创建一个新分支用于合并开发者的更改,再通过git push语句将开发者的修改发送到远程仓库。
\subsubsection{解题感悟}
在使用git的过程中注意不要覆盖别人的代码,要新建分支。
\subsection{实例9}
\subsubsection{练习内容}
使用git下载指定版本代码
\subsubsection{结果}
首先使用git clone+地址,进入克隆的文件夹使用cd +文件名,然后查看仓库的历史提交记录,找到想要下载的版本的提交哈希值,使用git checkout+该值切换到指定版本,再使用git log指令确认切换成功。
\subsubsection{解题感悟}
通过这种操作用户可以随心选择以前的版本代码,便于用户选择,满足用户需求。
\subsection{实例10}
\subsubsection{练习内容}
解决git冲突
\subsubsection{结果}
首先使用git status命令查看冲突的文件,然后取舍保留哪些更改,并删除冲突标记,再使用git add+文件名命令以及git commit指令合并文件。
\subsubsection{解题感悟}
git的这种冲突有利于团队合作,防止代码混乱变换。
\section{LaTeX}
\subsection{实例1}
\subsubsection{练习内容}
将标题、姓名、学号、日期加粗并加下划线
\subsubsection{结果}
如果直接对maketitle语句进行操作会使首页变为正文页,而第二页变为原来的首页,不符合预期,经过尝试,发现应该在title、author、date语句中进行操作
\subsubsection{解题感悟}
经过这次练习,我明白了maketitle是功能性语句,而如果想要修改标题的内容和格式应该对title、author、date这些信息语句进行修改。
\subsection{实例2}
\subsubsection{练习内容}
在代码下面添加图片并规定图片大小为0.6倍页面宽度,并使图片居中
\subsubsection{结果}
首先将图片添加到main.tex文件的下面,然后通过begin\ {figure}环境中的includegraphics[]{}语句中添加图片文件命名,在[]中输入width=0.6textwidth即可将图片调节为0.6倍页面宽度,在begin\ {figure}后加上[h],就可以使图片出现在代码下面,centering语句使图片居中。
\subsubsection{解题感悟}
LaTeX中添加图片很方便,同时还可以规范图片格式,数学论文中可以使用,比其他软件优越。
\subsection{实例3}
\subsubsection{练习内容}
并排添加图片
\subsubsection{结果}
通过在begin\ {figure}环境中添加begin\ {minipage}环境在这个环境中就可以并排添加图片了,并且在minipage后添加width=0.4textwidth就可以将每个图片的页面大小调整到整个页面的0.4倍,然后图片的width=
0.6textwidth并不会超出页面,而是在0.4基础上的0.6也就是0.24倍总页面宽度。
\subsubsection{解题感悟}
LaTeX优于Word的地方之一就是可以并排添加图片,同时可以规范图片大小,极为方便
\subsection{实例4}
\subsubsection{练习内容}
让页数自动清点总页数并展示出来
\subsubsection{结果}
通过pageref语句中输出LastPage页,但只是这样编译后会出现两个?,这时应该加入宏宝lastpage就可以正常显示了。
\subsubsection{解题感悟}
LaTeX通过代码来编译为文本使得软件非常智能,无需亲自清点页数,非常省力。
\subsection{实例5}
\subsubsection{练习内容}
人工输入列表修改时特别繁琐,如何智能列表并在列表中添加内容时自动编号
\subsubsection{结果}
通过在begin\ {enumerate}环境中添加\ item语句,然后在item后面书写内容,要修改时直接在中间填入一行item语句,系统自动修改编号,无需手动修改。
\subsubsection{解题感悟}
LaTeX的列表功能不仅对仗工整,每个序号都上下对齐而且自动编号,省时省力。
\subsection{实例6}
\subsubsection{练习内容}
让标题独占一页
\subsubsection{结果}
通过在maketitle语句下方加入newpage语句就可以使标题独占一页
\subsubsection{解题感悟}
这个功能使得整个文本的格式清晰合理。
\subsection{实例7}
\subsubsection{练习内容}
自定义页眉
\subsubsection{结果}
文本中有6个页眉区域分别是上左,上中、上右、下左、下中、下右,想要自定义页眉,首先引入fancyhdr宏包,然后使用fancyhf,lhead表示上左,chead表示上中、rhead表示上右、lfoot表示下左、cfoot表示下中、rfoot表示下右,在这6个语句中添加自定义的页眉内容即可。
\subsubsection{解题感悟}
通过这种方式自定义的页眉工整,位置恰当,看起来不突兀,很美观。
\subsection{实例8}
\subsubsection{练习内容}
为文本添加目录
\subsubsection{结果}
首先注意的一点,系统是以小标题来设计目录的,所以书写标题时要用section、subsection、subsubsection语句,然后通过tableofcontents语句,系统自动规划好目录并呈现给使用者。
\subsubsection{解题感悟}
如果人工输入目录,不仅繁琐而且修改复杂,而让系统自动生成就方便许多。
\subsection{实例9}
\subsubsection{练习内容}
设置页面的布局即页面的参数
\subsubsection{结果}
首先引入geometry宏包,然后使用geometry语句,
在语句中书写比如left=2cm,right=2cm,top=3.5cm,
bottom=3.5cm,参数可以随意更改。
\subsubsection{解题感悟}
通过设置页面参数可以使整个文本看起来更加美观,赏心悦目。
\subsection{实例10}
\subsubsection{练习内容}
让系统自动输出当前页数及当前日期
\subsubsection{结果}
文章中常常在下中页眉处表明当前页数,但系统默认输出一个数字,但如果你想输出“第几页”,就需要运用cfoot语句的同时用到thepage语句,这个语句可以自动输出当前页数比如“第\ thepage 页”,要注意的是thepage和旁边文字间必须有空格,否则会出错导致无法输出当前页数。

如果要输出当前日期,只需要用到today语句就可以了。
\subsubsection{解题感悟}
LaTeX的这些操作减少了使用者的修改工作,大大方便了使用者使用。
\end{document}
